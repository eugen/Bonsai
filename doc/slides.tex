\documentclass{beamer}
\usepackage[romanian]{babel}
%% Antibes Bergen Berkeley Berlin
%% Boadilla Copenhagen Darmstadt Dresden
%% Frankfurt Goettingen Hannover Ilmenau
%% Juanlespins Madrid Malmoe Marburg
%% Montpellier Paloalto Pittsburgh Rochester
%% Singapore
\usetheme{Dresden}
\usepackage{graphicx}
\usepackage{tikz}
\usepackage{fontspec}
\usepackage{verbatim}

\setmonofont{Inconsolata}
\setsansfont{Calibri}

\setbeamertemplate{navigation symbols}{}

\renewcommand{\c}[1]{\textcolor{blue}{\texttt{#1}}}

\title[Bonsai - un limbaj funcțional peste DLR]{Bonsai}
\subtitle{Un limbaj funcțional peste DLR}
\author{Eugen Anghel}
\institute{Coordonator științific\\Profesor Doctor Gheorghe Grigoraș}
\date{\today}
\begin{document}

\frame{\titlepage}

\begin{frame}
  \frametitle{Cuprins}
  \tableofcontents
\end{frame}

\begin{frame}
  \frametitle{Introducere}
  Bonsai este un limbaj:
  \begin{itemize}
  \item 
    Dinamic
    \begin{itemize}
      \item apelurile sunt rezolvate la rulare
    \end{itemize}
  \item 
    Funcțional
    \begin{itemize}
    \item orice expresie este un apel de funcție
    \item orice instanță poate fi apelată
    \end{itemize}
  \item 
    OOP
    \begin{itemize}
    \item programare orientată-obiect bazată pe prototipuri
    \end{itemize}
  \item
    Posibilitatea de a utiliza biblioteci .NET
  \item
    Simplu
    \begin{itemize}
    \item fără cuvinte cheie
    \item fără (prea multe) construcții speciale
    \item gramatică minimală
    \end{itemize}
  \end{itemize}
\end{frame}

\section{Tehnologii utilizate}
\begin{frame}
  \frametitle{Tehnologii utilizate}
  \begin{itemize}
    \item 
      Antlr 3.1
      \begin{itemize}
      \item Parsarea codului
      \item Generarea AST-ului din arborele de parsare
      \end{itemize}
    \item
      .NET Framework 4.0 Beta 2
      \begin{itemize}
        \item C\# 4.0
        \item Dynamic Language Runtime
      \end{itemize}
    \item
      Visual Studio 2010 Beta 2
      \begin{itemize}
        \item Definirea și rularea testelor automate
      \end{itemize}
  \end{itemize}
\end{frame}

\section{Runtime}
\begin{frame}
  \frametitle{Runtime-ul}
  Runtime-ul Bonsai
  \begin{itemize}
  \item AST -> expresii LINQ
    \begin{itemize}
      \item Tratarea contextului
    \end{itemize}
  \item Compilarea expresiilor
    \begin{itemize}
      \item Dynamic Language Runtime
    \end{itemize}
  \item Clasa \c{BonsaiBinder}
    \begin{itemize}
      \item rezolvarea apelurilor
      \item integrarea cu platforma .NET
    \end{itemize}
  \end{itemize}
\end{frame}

\section{Sintaxa}
\begin{frame}
  \frametitle{Sintaxa}
  Elemente de sintaxă:
  \begin{itemize}
  \item identificatori: \c{name}, \c{name\_2}, \c{+}
  \item simboluri: \c{.name}, \c{.name\_2}, \c{.+}
  \item numere: \c{7}, \c{0.009}, \c{-1000.0001}
  \item string-uri: \c{'sir 1'}, \c{"sir 2"}
  \item paranteze: \c{((a .+ b) ./ (c .+ d))}
  \item structuri de date: \c{[| 1 2 3 ]}
  \item blocuri: \c{\{ user .call \}}, \c{\{ print "done" \}}
  \end{itemize}
\end{frame}

\begin{frame}[containsverbatim]
  \frametitle{Sintaxa (2)}
  Orice expresie este un apel de funcție, inclusiv operațiile matematice, definirea funcțiilor și instrucțiunile condiționale:
  \begin{itemize}
    \item
      apel al funcției \c{3}:
      \color{blue}
      \begin{verbatim}
  3 .+ 2      \end{verbatim}
      \color{black}
    \item
      apel al funcției \c{defun}:
      \color{blue}
      \begin{verbatim}
  defun .f .a .b { 
    a .+ b
  }  \end{verbatim}
      \color{black}
    \item
      apel al funcției \c{if}:
      \color{blue}
      \begin{verbatim}
  if (== a 0) { 
    print "done"
  } {
    print "a > 0"
  }      \end{verbatim}
      \color{black}

  \end{itemize}
\end{frame}

\section{Funcții}
\begin{frame}
  \frametitle{Funcții standard}
  Funcții standard în Bonsai
  \begin{itemize}
    \item \c{=} și \c{=:}: asignarea variabilelor
    \item \c{==}, \c{>}, \c{<}, etc: comparare 
    \item \c{if}, \c{when} și \c{unless}: funcții condiționale
    \item \c{ref}: referențierea unei funcții
    \item \c{import}: importul unui spațiu de nume sau al unei clase dintr-o bibliotecă .NET
    \item \c{print}: afișează argumentele în consolă
    \item \c{null}: întoarce valoarea null
    \item etc.
  \end{itemize}
\end{frame}

\begin{frame}[containsverbatim]
  \frametitle{Definirea funcțiilor}
  Funcția \c{defun} crează o funcție
  \begin{itemize}
  \item
    \color{blue}
    \begin{verbatim}
  defun .sqr .n { n .* n }\end{verbatim}
  \item
    \begin{verbatim}
  defun .hypotenuse .c1 .c2 { 
    Math .sqrt (sqr c1) (sqr c2) }\end{verbatim}
    \color{black}
  \end{itemize}
  \vskip8pt
  Funcția \c{def|} definește o nouă variantă a unei funcții pentru un set de șabloane
  \begin{itemize}
  \item șabloanele sunt create prin apelul funcțiilor \c{|=}, \c{|is}, etc.
  \item simboluri — șabloane care se potrivesc peste orice valoare
  \item exemplu de utilizare:
    \color{blue}
    \begin{verbatim}
  def| .factorial (|= .zero 0) { 1 }
  def| .factorial .a { a .* (factorial (a .- 1)) }\end{verbatim}
    \color{black}
  \end{itemize}
\end{frame}

\section{Structuri de date}
\begin{frame}[containsverbatim]
  \frametitle{Structuri de date declarative}
  Mecanism extensibil de declarare a structurilor de date
  \begin{itemize}
    \item Liste
      \begin{itemize}
        \item \c{[| 1 2 3 ]}
        \item \c{|| "a" "b" "c" ]}
      \end{itemize}
    \item Dicționare
      \begin{itemize}
        \item \c{[\# .nume "Popescu" .prenume "Ion" ]}
      \end{itemize}
    \item Se poate defini o sintaxă pentru structuri noi de date
      \color{blue}
      \begin{verbatim}
  defun .rgbDataHandler .r .g .b {
    Color .FromArgb r g b }
  dataHandlers .Add .rgb rgbDataHandler
  = .white [rgb 255 255 255 ]
  = .red [rgb 255 0 0 ]
      \end{verbatim}
  \end{itemize}
\end{frame}

\section{Programarea orientată obiect}
\begin{frame}
  \frametitle{Programarea orientată obiect}
  Programare bazată pe prototipuri
  \begin{itemize}
    \item orice obiect nou se obține prin clonarea unui alt obiect (prototip)
    \item unui obiect i se pot adăuga metode și câmpuri, care vor fi moștenite de obiectele clonate din el
    \item conceptul de "clasă" nu este folosit
  \end{itemize}
  \vskip8pt
  Modificatorii de acces nu sunt definiți
  \begin{itemize}
    \item prin convenție, metodele care încep cu caracterul \c{\_} sunt considerate private
  \end{itemize}
  \vskip8pt
  Polimorfismul este prezent implicit (limbaj dinamic)
\end{frame}

\begin{frame}[containsverbatim]
  \frametitle{Programarea orientată obiect (exemplu)}
  \color{blue}
  \begin{verbatim}
  # crează prototipul RectangleProto cu metoda area
  = .RectangleProto (object .clone)
  RectangleProto .method .area { 
    (self .height) .* (self .width) }
  # creaza o clona a RectangleProto
  = .rectangle1 (RectangleProto .clone)
  # defineste campurile width si height
  rectangle1 .field .width 4
  rectangle1 .field .height 5
  # acceseaza campul width
  rectangle1 .width                        -> 5
  # apeleaza metoda area
  rectangle1 .area                         -> 20\end{verbatim}
  \color{black}

\end{frame}

\section{Exemple}
\begin{frame}[containsverbatim]
  \frametitle{Un exemplu de program}
  \color{blue}
  \begin{verbatim}
  defun .make_counter .start {
    defun .add .increment {
      = .start (start .+ increment)
      start
    }
    ref .add
  }
  = .c1 (make_counter 0)
  = .c2 (make_counter 10)
  print (c1 2)                         -> 2
  print (c2 5)                         -> 15
  print (c1 2)                         -> 4
  print (c2 5)                         -> 20
  \end{verbatim}
  \color{black}
\end{frame}

\end{document}
