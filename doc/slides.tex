\documentclass{beamer}
\usepackage[romanian]{babel}
\usetheme{Copenhagen}
%% \usecolortheme{beaver}
\usepackage{beamerthemesplit}
\usepackage{graphicx}
\usepackage{tikz}
\usepackage{fontspec}
\usepackage{verbatim}

\setmonofont{Inconsolata}
\setsansfont{Calibri}

\setbeamertemplate{navigation symbols}{}

\renewcommand{\c}[1]{\textcolor{blue}{\texttt{#1}}}

\title[Bonsai - un limbaj funcțional peste DLR]{Bonsai}
\subtitle{Un limbaj funcțional peste DLR}
\author{Eugen Anghel}
\institute{Coordonator științific\\Profesor Doctor Gheorghe Grigoraș}
\date{\today}
\begin{document}

\frame{\titlepage}

\begin{frame}
  \frametitle{Introducere}
  Bonsai este un limbaj:
  \begin{itemize}
  \item 
    Funcțional
    \begin{itemize}
    \item orice expresie este un apel de funcție
    \item orice instanță poate fi apelată
    \end{itemize}
  \item 
    OOP
    \begin{itemize}
    \item programare orientată-obiect bazată pe prototipuri
    \end{itemize}
  \item
    Posibilitatea de a utiliza biblioteci .NET
  \item
    Simplu
    \begin{itemize}
    \item fără cuvinte cheie
    \item fără (prea multe) construcții speciale
    \item gramatică minimală
    \end{itemize}
  \end{itemize}
\end{frame}

\begin{frame}
  \frametitle{Tehnologii utilizate}
  \begin{itemize}
    \item 
      Antlr 3.1
      \begin{itemize}
      \item Parsarea codului
      \item Generarea AST-ului din arborele de parsare
      \end{itemize}
    \item
      .NET Framework 4.0 Beta 2
      \begin{itemize}
        \item C\# 4.0
        \item Dynamic Language Runtime
      \end{itemize}
    \item
      Visual Studio 2010 Beta 2
      \begin{itemize}
        \item Definirea și rularea testelor automate
      \end{itemize}
  \end{itemize}
\end{frame}

\begin{frame}
  \frametitle{Sintaxa}
  Număr redus de elemente de sintaxă:
  \begin{itemize}
  \item identificatori: \c{name}, \c{name\_2}, \c{+}
  \item simboluri: \c{.name}, \c{.name\_2}, \c{.+}
  \item numere: \c{7}, \c{0.009}, \c{-1000.0001}
  \item string-uri: \c{'sir 1'}, \c{"sir 2"}
  \item paranteze: \c{((a .+ b) ./ (c .+ d))}
  \item structuri de date: \c{[| 1 2 3 ]}
  \item blocuri: \c{\{ user .call \}}, \c{\{ print "done" \}}
  \end{itemize}
  \vskip8pt
  Orice expresie este un apel de funcție, inclusiv operațiile matematice, definirea funcțiilor și instrucțiunile condiționale
\end{frame}

\begin{frame}[containsverbatim]
  \frametitle{Sintaxa - 2}
  Orice expresie este un apel de funcție, inclusiv operațiile matematice, definirea funcțiilor și instrucțiunile condiționale:
  \begin{itemize}
    \item
      apel al funcției \c{3}:
      \color{blue}
      \begin{verbatim}
  3 .+ 2      \end{verbatim}
      \color{black}
    \item
      apel al funcției \c{defun}:
      \color{blue}
      \begin{verbatim}
  defun .f .a .b { 
    a .+ b
  }  \end{verbatim}
      \color{black}
    \item
      apel al funcției \c{if}:
      \color{blue}
      \begin{verbatim}
  if (== a 0) { 
    print "done"
  } {
    print "a > 0"
  }      \end{verbatim}
      \color{black}

  \end{itemize}
\end{frame}

\begin{frame}[containsverbatim]
  \frametitle{Sintaxa - 3}
  Apelurile sunt separate de unul din caracterele \c{'\textbackslash n'} sau \c{';'}.
  \begin{itemize}
    \item 
      codul \\
      \c{\ \ print a; print b}
    \item 
      este echivalent cu \\
      \c{\ \ print a}\\
      \c{\ \ print b}
  \end{itemize}
  Pentru a apela o funcție cu argumente pe rânduri diferite se pot utiliza parantezele. 
\begin{itemize}
\item codul \c{f (a .+ b) (c .+ d)}
\item este echivalent cu\\
\c{\ \ f (}\\
\c{\ \ \ \ a .+ b ) (}\\
\c{\ \ \ \ c .+ d )}
\end{itemize}
\end{frame}

\begin{frame}[containsverbatim]
  \frametitle{Definirea funcțiilor}
  Funcția \c{defun}
  \begin{itemize}
    \item are antetul \c{defun .nume (.nume\_parametru)* \{block\}}
    \item crează o funcție și o salvează în contextul curent, în variabila cu numele \c{nume}
    \item \c{(.nume\_parametru)*} este o înșiruire de simboluri care conține numele parametrilor funcției
    \item exemple de utilizare:
      \begin{itemize}
      \item
        \color{blue}
        \begin{verbatim}
  defun .sqr .n { n .* n }\end{verbatim}
      \item
        \begin{verbatim}
  defun .hypotenuse .c1 .c2 { 
    Math .sqrt (sqr c1) (sqr c2) }\end{verbatim}
        \color{black}
      \end{itemize}
    \item dacă variabila există deja atunci aceasta este suprascrisă
  \end{itemize}
\end{frame}

\begin{frame}[containsverbatim]
  \frametitle{Pattern matching}
  Funcția \c{def|}
  \begin{itemize}
    \item crează o funcție cu pattern-matching și o salveaza în variabila \c{name}
    \item are antetul \c{def| .nume (sablon)* \{block\}}
    \item șabloanele pot fi
      \begin{itemize}
        \item instanțe ale unor clase care implementează interfața \c{IPattern}
        \item simboluri — șabloane care se potrivesc peste orice valoare
      \end{itemize}
    \item funcții care întorc șabloane: \c{|=}, \c{|is}, etc.
    \item fiecare apel al \c{def|} definește o nouă variantă, un nou set de șabloane pentru o funcție
    \item exemplu de utilizare:
      \color{blue}
      \begin{verbatim}
  def| .factorial (|= .zero 0) { 1 }
  def| .factorial .a { a .* (factorial (a .- 1)) }\end{verbatim}
        \color{black}
  \end{itemize}
\end{frame}

\begin{frame}
  \frametitle{Structuri de date declarative}
  
\end{frame}

\begin{frame}[containsverbatim]
  \frametitle{Un exemplu de program}
  \color{blue}
  \begin{verbatim}
  defun .make_counter .start {
    defun .add .increment {
      = .start (start .+ increment)
      start
    }
    ref .add
  }
  = .c1 (make_counter 0)
  = .c2 (make_counter 10)
  print (c1 2)                         -> 2
  print (c2 5)                         -> 15
  print (c1 2)                         -> 4
  print (c2 5)                         -> 20
  \end{verbatim}
  \color{black}
\end{frame}

\begin{frame}
  \frametitle{Runtime-ul}
  
\end{frame}

\end{document}

- Exemplu de cod (ceva cu lexical scoping, functii, etc) + Explicații în 2-3 slides
- Toate operațiile sunt apeluri de funcții
- Operații matematice 

- Definirea funcțiilor

- Pattern matching

- Structuri de date declarative

- Programare orientată obiect

- Apelarea bibliotecilor .NET
- Îmbunătățiri
