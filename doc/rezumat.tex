\documentclass[12pt,a4paper]{memoir}
\usepackage[romanian]{babel}
\usepackage[dvipsnames]{color}
\usepackage{fancyvrb,fontspec,xunicode,xltxtra,hyperref,url,framed,verbatim}
% nicer fonts
\setmainfont{Calluna}
\setmonofont{Inconsolata}

\begin{document}
\thispagestyle{empty}
\begin{centering}
\textbf{\HUGE{Bonsai}\\\Large{Un limbaj dinamic funcțional peste DLR}}
\\[1.5cm]
\end{centering}
În ultimele doua decenii, programarea a fost în mare parte împărțită între două grupe de limbaje. Limbajele imperative, cu un sistem static de tipuri, multe dintre ele din familia C, au fost folosite pentru a obține performanță și scalabilitate. În schimb, flexibilitatea limbajelor precum Perl si PHP a fost apreciată pentru programarea scripturilor de sistem, a paginilor web și a altor programe pentru care eficiența nu reprezintă o prioritate.

În ultimii ani însă, odată cu reducerea exponențială a costurilor hardware-ului — mulțumită legii lui Moore — accentul a început să se pună din ce în ce mai puțin pe eficiență și din ce în ce mai mult pe timpul de dezvoltare. Performanța limbajelor dinamice a devenit „suficientă”, ceea ce a condus la o adevărată renaștere a acestora.

Atât Sun cât și Microsoft și-au arătat interesul pentru limbajele dinamice prin proiectele The Da Vinci Machine și Dynamic Language Runtime și sprijinirea dezvoltării limbajelor JRuby, Jython, IronRuby, IronPython, etc.

Lucrarea de față prezintă un limbaj dinamic numit Bonsai. Principalele sale caracteristici sunt:
\begin{itemize}
\item simplitate: gramatică minimală, zero cuvinte cheie
\item orice instanță este privită ca o funcție;
\item orice expresie este un apel de funcție;
\item funcțiile și blocurile de cod pot fi trimise ca argumente altor funcții, returnate sau stocate în variabile;
\item suportă o formă limitată de \emph{pattern-matching} peste parametrii funcțiilor
\item deține un mecanism extensibil de declarare a structurilor de date 
\item implementează o formă de programare orientată obiect bazată pe prototipuri
\item poate utiliza bibliotecile .NET.
\end{itemize}

Pe lângă documentarea semanticii limbajului, lucrarea descrie detalii de implementare și explică modul în care este utilizat Dynamic Language Runtime.

\end{document}
