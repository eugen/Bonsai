\documentclass[12pt,a4paper]{memoir}
\usepackage{url}
\usepackage{fancyvrb}
\usepackage{fontspec} 
\usepackage{xunicode}
\usepackage[dvipsnames]{color}
\usepackage{xltxtra}
\usepackage[romanian]{babel}

\hyphenpenalty=5000
\tolerance=1000

\setmainfont{Calluna}
\setmonofont{Inconsolata}

\fvset{numbers=left,labelposition=all,frame=lines}

\definecolor{shadecolor}{named}{Gray}

\title{Bonsai\\Un limbal dinamic funcțional peste DLR}
\author{Eugen Anghel}

\begin{document}

\newcommand{\draft}{\textcolor{red}}

\maketitle

\newpage
\tableofcontents

%%%%%%%%%%%%%%%%%%%%%%%%%%%%%%%%%%%%%%%%%%%%%%%%%%
\chapter{Introducere}
%%%%%%%%%%%%%%%%%%%%%%%%%%%%%%%%%%%%%%%%%%%%%%%%%%

În ultimele doua decenii, programarea a fost în mare parte împărțită între două grupe de limbaje. Limbajele imperative, cu un sistem static de tipuri, multe dintre ele din familia C, au fost folosite pentru a obține performanță și scalabilitate. În schimb, flexibilitatea limbajelor precum Perl si PHP a fost apreciată pentru programarea scripturilor de sistem, a paginilor web și a altor programe pentru care eficiența nu reprezintă o prioritate.

În ultimii ani însă, odată cu reducerea exponențială a costurilor hardware-ului — mulțumita legii lui Moore — accentul a început să se pună din ce în ce mai puțin pe eficiență și din ce în ce mai mult pe timpul de dezvoltare. Performanța limbajelor dinamice a devenit „suficientă”, ceea ce a condus la o adevărată renaștere a acestora.

Atât Sun cât și Microsoft și-au arătat interesul pentru limbajele dinamice prin proiectele The Da Vinci Machine\cite{daVinciMachine} și Dynamic Language Runtime (DLR)\cite{dlr} și sprijinirea dezvoltării limbajelor JRuby\cite{jruby}, Jython\cite{jython}, IronRuby\cite{ironruby}, IronPython\cite{ironpython}, etc.

Lucrarea de față prezintă un limbaj dinamic numit Bonsai. Cea mai importantă caracteristică a sa este simplitatea, limbajele Lisp, Javascript și Io \cite{io} reprezentând principalele surse de inspirație. Platforma aleasă este .NET Framework, împreună cu Dynamic Language Runtime. Întreaga suită de biblioteci .NET este astfel disponibilă, indiferent de limbajul in care au fost ele scrise. Bonsai este un limbaj funcțional, dar se integrează bine cu alte limbaje orientate obiect: OOP-ul este simulat în întregime prin apeluri de funcții. Datorită platformei pe care rulează și a utilizării de biblioteci scrise in alte limbaje, transparența referențială a funcțiilor nu poate fi garantată.
\draft{Programarea fără efecte laterale este în schimb încurajată: majoritatea primitivelor din Bonsai sunt funcții pure, iar cele care modifica starea sunt clar denumite astfel încât acest lucru să-i fie vizibil programatorului}.

%%%%%%%%%%%%%%%%%%%%%%%%%%%%%%%%%%%%%%%%%%%%%%%%%%
\chapter{Gramatica și sintaxa limbajului}
%%%%%%%%%%%%%%%%%%%%%%%%%%%%%%%%%%%%%%%%%%%%%%%%%%

Simplitatea limbajului se reflectă și în dimensiunea gramaticii: aceasta conține doar 18 reguli de parsare si nici un cuvânt rezervat.

\VerbatimInput[label=Bonsai.g]{../src/Grammars/Bonsai.g}

Pe lângă gramatica de parsare, Bonsai mai utilizează o gramatica de transformare a arborelui de parsare de doar 3 reguli.

\VerbatimInput[label=BonsaiTree.g]{../src/Grammars/BonsaiTree.g}

%%%%%%%%%%%%%%%%%%%%%%%%%%%%%%%%%%%%%%%%%%%%%%%%%%
\chapter{Compilarea codului sursă}
%%%%%%%%%%%%%%%%%%%%%%%%%%%%%%%%%%%%%%%%%%%%%%%%%%

\draft{parsare}

\draft{AST}

\draft{Generator.cs -> Linq Expressions}

%%%%%%%%%%%%%%%%%%%%%%%%%%%%%%%%%%%%%%%%%%%%%%%%%%
\chapter{Runtime-ul Bonsai}
%%%%%%%%%%%%%%%%%%%%%%%%%%%%%%%%%%%%%%%%%%%%%%%%%%

\section{BonsaiBinder}

\section{Tratarea tipurilor de date primitive}

\section{Rezolvarea metodelor .NET}

\section{Tipul BonsaiFunction}

%%%%%%%%%%%%%%%%%%%%%%%%%%%%%%%%%%%%%%%%%%%%%%%%%%
\chapter{Biblioteca de funcții standard}
%%%%%%%%%%%%%%%%%%%%%%%%%%%%%%%%%%%%%%%%%%%%%%%%%%

Spre deosebire de proiecte precum IronPython si IronRuby, care au fost nevoite sa implementeze întreaga biblioteca disponibilă în implementarile „originale” (MRI\cite{ruby_mri} si CPython\cite{cpython}), Bonsai încearcă să folosească cât mai mult clasele din .NET Framework și să mențină biblioteca proprie de funcții cât mai mică.

\section{Numere}

În Bonsai există un singur tip de numere: raționale, reprezentate in virgulă fixă. Acesta mapează peste tipul \texttt{System.Decimal} din .NET Framework. Erorile cauzate de reprezentarea in virgulă mobilă (ex. \texttt{1.1 + 1.1 + 1.1 == 3.300000000007}) sunt astfel eliminate.

\section{Șiruri de caractere}

\section{Symboluri}

Tipul \texttt{symbol} din Bonsai este foarte asemănător simbolurilor din Ruby\cite{ruby_symbols} și Lisp\cite{clhs_symbols}. Un simbol se comportă de cele mai multe ori ca un string. Diferența majoră constă în faptul că pentru orice string există o singură instanță a unui simbol cu acel string, ceea ce transformă compararea a două simboluri         într-o comparație de pointeri, foarte rapidă. De asemenea, utilizarea unui simbol indică de cele mai multe ori folosirea metaprogrămarii. 

\section{Blocuri}

Un \texttt{block} este un set de instrucțiuni împreună cu o listă de variabile locale definite in el. Un block poate fi evaluat direct, dar și analizat sau modificat la runtime, pentru a obține un alt block.

\section{Funcții}

\section{Pattern Matching}

\section{Structuri de date}

\draft{Funcția \texttt{list} crează o listă. Majoritatea apelurilor către o listă întorc (unde este cazul) o instanță nouă, fără să modifice lista originală. Toate operațiile pe liste binecunoscute în limbajele funcționale (\texttt{map, reduce, foldl, foldr,} etc) sunt disponibile și în Bonsai.}

\section{Funcții de ordin superior}

\subsection{combine}

\subsection{apply}

\subsection{papply}

\subsection{ref}

%%%%%%%%%%%%%%%%%%%%%%%%%%%%%%%%%%%%%%%%%%%%%%%%%%
\chapter{Bonsai și Programarea Orientată Obiect}
%%%%%%%%%%%%%%%%%%%%%%%%%%%%%%%%%%%%%%%%%%%%%%%%%%

Spre deosebire de limbajele Io\cite{io} și SmallTalk\cite{smalltalk}, unde orice este un obiect, în Bonsai orice este o funcție. Mai mult, dacă în SmallTalk și Io orice expresie reprezintă trimiterea unui mesaj către un obiect, iar rezultatul tot un obiect, în Bonsai orice expresie este un apel de funcție, iar rezultatul tot o funcție.

Această diferență poate părea fundamentală, dar o funcție poate fi considerată generalizarea unui obiect. În secțiunile următoare vom vedea cum toate cele trei principii fundamentale OOP – moștenirea, encapsularea și polimorfismul – sunt „simulate” în Bonsai cu ajutorul funcțiilor.

\section{Paralela funcțional <-> OOP}

Baza programarii orientate obiect o reprezinta message-passing-ul. Acesta este emulat în Bonsai printr-un apel de funcție în care primul parametru este un simbol. Funcția reprezintă obiectul, iar simbolul, mesajul trimis. Spre exemplu, un string apelat cu simbolul \texttt{:upcase} va returna un nou string cu toate caracterele transformate în majuscule:
\begin{Verbatim}
"bonsai!" .upcase
\end{Verbatim}
Acest cod are același efect cu următorul cod C\#:
\begin{Verbatim}
"bonsai".ToUpper()
\end{Verbatim}
În cazul în care mesajul are mai mulți parametri, aceștia urmează după simbol:
\begin{Verbatim}
stringBuilder .Append " ultima parte "
\end{Verbatim}

\begin{shaded}
zomg wtf
\end{shaded}


Moștenirea poate fi simulată printr-o funcție care este definită doar pentru un set de pattern-uri peste parametri, apelurile cu alți parametri fiind redirectate către o alta funcție (analogul clasei de bază).

Tratarea mesajelor în acest fel poate părea primitivă și greoaie, dar în practică nu se face niciodată o tratare manuală (de tip \texttt{case}): toate operațiile OOP „clasice” sunt abstractizate printr-un set de funcțtii incluse în biblioteca standard a limbajului.

\section{Creerea prototipurilor}

\section{Clonarea instanțelor noi din prototipuri}

\draft{Metoda \texttt{clone}}

\section{Encapsularea}

\draft{Dicționarele \texttt{private} și \texttt{protected}}

\section{Polimormism?}

\draft{Limbaj dinamic = polimorfism implicit (?)}

%%%%%%%%%%%%%%%%%%%%%%%%%%%%%%%%%%%%%%%%%%%%%%%%%%
\chapter{Integrarea cu platforma .NET}
%%%%%%%%%%%%%%%%%%%%%%%%%%%%%%%%%%%%%%%%%%%%%%%%%%

\section{primitiva \texttt{import}}

\section{Spații de nume și clase statice}

\section{Instanțierea obiectelor}

\section{Pasarea funcțiilor Bonsai către metode .NET}

\section{Integrarea cu alte limbaje dinamice}

%%%%%%%%%%%%%%%%%%%%%%%%%%%%%%%%%%%%%%%%%%%%%%%%%%
\chapter{Părți lipsă și îmbunătățiri pentru viitor}
%%%%%%%%%%%%%%%%%%%%%%%%%%%%%%%%%%%%%%%%%%%%%%%%%%

\section{Module}

\section{Spații de nume}

\section{Metaprogramare}

\draft{inspectarea codului sursă al funcțiilor la rulare; modificarea și recompilarea acestuia}

\draft{similar, modificarea și recompilarea AST-ului la rulare}

\draft{posibilitatea evaluării non-stricte a argumentelor (à la Haskell); avantaje/dezavantaje}

\section{Debugger}

\draft{posibilități de realizare a unui debugger; integrarea cu VisualStudio}

\section{Îmbunătățirea performanței}

\section{Tratarea excepțiilor}


%%%%%%%%%%%%%%%%%%%%%%%%%%%%%%%%%%%%%%%%%%%%%%%%%%
\bibliography{bonsai}{}
\bibliographystyle{plain}

\end{document}
